% ===========================================================================
% sections.tex
% ============---------------------------------------------------------------
%
% (C) in 2017-18 by Norman Markgraf (nmarkgraf(at)hotmail(dot)com)
%
% ---------------------------------------------------------------------------
%
% 13.07.2018 
%
% ---------------------------------------------------------------------------
%
\def\imagepath{images/}
\def\backgroundimage{\imagepath./hintergrund}
\def\lineimage{\imagepath./linie}
\def\logoimage{\imagepath./logo}
\def\titlebackgroundimage{\imagepath./titlehintergrund}
%
\def\logowidth{1cm}
\def\logoheight{1cm}
\def\logoxshift{-0.7cm}
\def\logoyshift{-0.7cm}
%
\def\titleboxwidth{0.5\linewidth}
\def\sectiontitleboxwidth{0.75\linewidth}
\def\sectionpagewidth{0.5\myPaperWidth}
\def\sectionpageheight{0.25\myPaperHeight}
\def\subsectionpagewidth{0.5\myPaperWidth}
\def\subsectionpageheight{0.25\myPaperHeight}
%
\def\myleftskip{0.4cm}
\def\myrightskip{0.0cm}
\def\mysepsize{0.6em}

% My FrameTitleBox ...
\newsavebox{\myftbox}
\newlength{\myftboxwidth}
\newlength{\myftboxmaxwidth}
\setlength{\myftboxmaxwidth}{390pt}
\def\myftboxscale{1.0}
\newlength{\mytitlebackgroundimageheight}

% ---------------------------------------------------------------------------

% ---------------------------------------------------------------------------
\usepackage{ragged2e} % Liefert FlushRight-Umgebung
% ---------------------------------------------------------------------------
\usepackage{ifthen}
\usepackage{tikz}
\usetikzlibrary{positioning}
\usetikzlibrary{calc}
\usetikzlibrary{backgrounds}
\usepackage{pgfplots}
\pgfplotsset{compat=1.3}
\usepackage{xspace}
\usepackage{array}
\usepackage{xstring}

\ifstrempty{\insertinstitute}{%
  \institute{FOM}%
}{\relax}

% ---------------------------------------------------------------------------
%
% Übungen und offene Übungen haben bei der FOM einen etwas anderen 
% Hintergrund beim Frametitle. Hier wird ein Schalter dazu erzeugt und
% und auf die normale Darstellun (false) gesetzt.
%
\newif\ifnpbtexercisemode
\npbtexercisemodefalse
%  \npbtexercisemodetrue



% ---------------------------------------------------------------------------
%
% Die sich regelmässig wiederholenden Bildelemente werden in pgf gespeichert
% und wiederverwendet. Dazu werden die folgenden Bezeichner verwendet:
%
% npbtlogo                :
% npbtlogoline            :
% npbtalternativelogoline :
%
% Mit dem Befehl "\updateNPBTimages" werden diese Bezeichner auf die aktuellen
% Werte gesetzt, falls diese durch themes geändert werden/wurden.
%
\newcommand*{\updateNPBTimages}{%
  \pgfdeclareimage[width=1.4\logowidth,height=1.4\logoheight]{npbtlogo}{\logoimage}
  \pgfdeclareimage[width=0.86\paperwidth]{npbtlogoline}{\lineimage}
  \pgfdeclareimage[width=0.86\paperwidth, height=1.80em]{npbtalternativelogoline}{\titlebackgroundimage}
}

\updateNPBTimages

% ---------------------------------------------------------------------------
%
% logoline
%

\newcommand{\logoline}{%
  \pgfuseimage{npbtlogoline}
}

\newcommand{\putLogo}{%
  \node[%
    shift={(\logoxshift,\logoyshift)},
    inner sep=0pt
  ] at (current page.north east){%
      \pgfuseimage{npbtlogo}
  };%
}

\newcommand{\setLogo}{%
  \begin{tikzpicture}[%
      remember picture, 
      overlay
    ]
      \putLogo%
  \end{tikzpicture}%
}

% ---------------------------------------------------------------------------
%
% title page:
%
\defbeamertemplate{title page}{minimal}
{
  \inserttitle
  \par
  
  \insertinstitute ~\xspace -- \insertdate
  \par
  
  \insertauthor
  \par
}

\defbeamertemplate{title page}{normal}
{
  \begin{tikzpicture}[remember picture, overlay]
    \node[inner sep=0pt] at (current page.center){%
              \includegraphics[width=\myPaperWidth,height=\myPaperHeight]{\backgroundimage}%
      };%
    \putLogo%
    \node[%
      anchor=south east,
      xshift=-1.95cm,
      yshift=3.65cm
    ] at (current page.south east){%
        \begin{minipage}[b]{\titleboxwidth}
            \begin{flushright} % XXXX
                {\usebeamercolor[fg]{title in titlepage}%
                 \usebeamerfont{title in titlepage} \inserttitle}
                \par
                \vspace*{0.7em}
                {\usebeamercolor[fg]{subtitle in titlepage}%
                 \usebeamerfont{subtitle in titlepage} \insertsubtitle}
                 \par
                {\usebeamercolor[fg]{institute in titlepage}%
                 \usebeamerfont{institute in titlepage} \insertinstitute ~\xspace -- \insertdate}
                \usebeamerfont{institute in titlepage}\par
                {\usebeamercolor[fg]{author in titlepage}%
                 \usebeamerfont{author in titlepage} \insertauthor}
            \end{flushright} % XXXX
        \end{minipage}    
    };%
  \end{tikzpicture}%
}

% ---------------------------------------------------------------------------
%
% Frame title template:
%
\defbeamertemplate{headline}{sectioninhead}{%
  \nointerlineskip%
  \begin{beamercolorbox}[%
      %ht=3.00em, %3.25em, %3em sollten eigentlich etwa 1,29168ex sein
      ht=6.0ex,
      dp=1ex,
      sep=0.1pt, %
      leftskip=\myleftskip, %
      rightskip=\myrightskip, %
  ]{headline}
    {%
      \vspace*{3ex}% Keine Ahnung wieso, aber ...
      \vspace*{-4.5ex}% ... es hat den Effekt, den ich haben will.
      \usebeamercolor[fg]{section in head}%
      \usebeamerfont{section in head}%
      \thesection.~\NoHyper\insertsection\endNoHyper\phantom{X}%
    }%
  \end{beamercolorbox}%
  \nointerlineskip%
% Hier gehört die Kapitelbezeichnung rein!
}

\defbeamertemplate{headline}{nosectioninhead}{%
  \nointerlineskip%
  \begin{beamercolorbox}[%
      ht=6.0ex,
      dp=1ex,
      sep=0.1pt, %
      leftskip=\myleftskip, %
      rightskip=\myrightskip, %
  ]{headline}
  \phantom{X}
  \end{beamercolorbox}
}

\defbeamertemplate{headline}{minimal}{
\relax%
}

%
% ---------------------------------------------------------------------------
%
\defbeamertemplate{frametitle}{minimal}{%
  \usebeamercolor[fg]{frametitle in head}%
  \usebeamerfont{frametitle in head}%
  \insertframetitle%
}

\defbeamertemplate{frametitle}{normal}{%
  \npbtexercisemodefalse%
  \makeatletter%
  \expandarg%
  \IfSubStr*{\beamer@frametitle}{bung}{\npbtexercisemodetrue}{}%
  \IfBeginWith*{\beamer@frametitle}{Offene}{\npbtexercisemodetrue}{}%
  \makeatother%
  \savebox{\myftbox}{%
      \ifnpbtexercisemode\usebeamercolor[fg]{frametitle in head exercise}\else\usebeamercolor[fg]{frametitle in head}\fi%
      \ifnpbtexercisemode\usebeamerfont{frametitle in head exercise}\else\usebeamerfont{frametitle in head}\fi%
      \insertframetitle}% end of savebox!
  \ifnpbtexercisemode{%
    \vskip1pt%
    \begin{tikzpicture}[remember picture, overlay]
        \node[
            anchor=west, 
            inner sep=0pt, 
            yshift=-\logoheight+4pt,
            ] at (current page.north west){%
                \pgfuseimage{npbtalternativelogoline}
          };%
    \end{tikzpicture}%
    \vskip-43pt%
  }\else\fi% end of ifnpbtexercisemode
  \settowidth{\myftboxwidth}{\usebox{\myftbox}}%
  \vskip6pt\vskip0pt%
  \begin{beamercolorbox}[%
      ht=1.20em, %
      sep=0pt, %
      leftskip=\myleftskip, %
      rightskip=\myrightskip, %
      wd=\paperwidth %
    ]{frametitle}%
      \ifdim\myftboxwidth>\myftboxmaxwidth%
        \resizebox{\myftboxmaxwidth}{!}{\usebox{\myftbox}}\\[-0.85em]
      \else%
        \usebox{\myftbox}\\[-0.85em]
      \fi
      \ifnpbtexercisemode\else\logoline\fi%
  \end{beamercolorbox}%
  \begin{tikzpicture}[%
      remember picture, %
      overlay %
    ]
      \putLogo%
  \end{tikzpicture}%
  \vskip-1.05em\vskip0pt%
}

% ---------------------------------------------------------------------------
%
% Footline template
%
\defbeamertemplate{footline}{nofootline}{%
  \relax%
}

\defbeamertemplate{footline}{normal}{% Skip first frame == Title frame !!!
%    \leavevmode%
    \ifthenelse{\equal{\insertframenumber}{1}}{%
    }{%
      \begin{beamercolorbox}[%
          ht=0.08ex,
          leftskip=\myleftskip,
          rightskip=\myrightskip
        ]{footline}
        \logoline%
      \end{beamercolorbox}%
      \begin{beamercolorbox}[%
          ht=1.25ex, 
          %dp=2.25ex,
          dp=1.35ex,
          leftskip=\myleftskip,
          rightskip=\myrightskip
        ]{footline}
        {%
          {%
            \usebeamercolor{date in footline}%
            \usebeamerfont{date in footline}%
            \insertdate  %
          }%
          \hfill%
          {%
            {%
              \usebeamercolor{author in footline}%
              \usebeamerfont{author in footline}%
              \insertshortauthor  %
            }%
            {%
              \usebeamercolor{separator in footline}%
              \usebeamerfont{separator in footline}%
              \hspace{\mysepsize}$|$\hspace{\mysepsize}%
            }%
            {%
              \usebeamercolor{title in footline}%
              \usebeamerfont{title in footline}%
              \inserttitle  %
            }%
          }%
          \hfill%
          \vspace*{-0.9em}%
          \raggedleft{%
            \usebeamercolor{number in footline}%
            \usebeamerfont{number in footline}%
            \insertframenumber \hspace*{0.3cm}%
          }%
        }%
      \end{beamercolorbox}
      \vspace*{0.2cm}
    }
}

% ---------------------------------------------------------------------------
%
% Section page template
%
%\setbeamertemplate{section page}
%{
%    \begin{centering}
%    \begin{beamercolorbox}[sep=12pt,center]{section title}
%      \usebeamerfont{section title}\insertsection\par
%    \end{beamercolorbox}
%    \end{centering}
%}
% ********
% FIX ME: Orginal aus dem PPTX: 4,41cm von rechts, 10,23cm vob oben ist eine 3,47cm hohe und
% ======= 18,85cm breite Box die von rechts unten her gefüllt wird!
% ********

\defbeamertemplate{section page}{minimal}
{
  \vfill\insertsectionhead\vfill
}

\defbeamertemplate{section page}{normal}
{
  \begin{tikzpicture}[remember picture, overlay]
    \node[anchor=south east,xshift=-1cm,yshift=1.5cm] at (current page.south east){%
        \begin{minipage}[b][\sectionpageheight][t]{\sectionpagewidth}
          \begin{FlushRight}
              \usebeamercolor{sectionnumber title}%
              \usebeamerfont{sectionnumber title}\insertsectionnumber %
              \usebeamercolor{section title}%
              \usebeamerfont{section title}\insertsectionhead\par
          \end{FlushRight}
        \end{minipage}
    };%
  \end{tikzpicture}
}

% ---------------------------------------------------------------------------
%
% Subsection page template
%
\defbeamertemplate{subsection page}{minimal}
{
  \vfill\insertsubsectionhead\vfill
}

\defbeamertemplate{subsection page}{normal}
{
  \begin{tikzpicture}[remember picture, overlay]
    \node[anchor=south east,xshift=-1cm,yshift=1.5cm] at (current page.south east){%
        \begin{minipage}[b][\sectionpageheight][t]{\sectionpagewidth}
          \begin{FlushRight}
              \usebeamercolor{sectionnumber title}%
              \usebeamerfont{sectionnumber title}\insertsubsectionnumber %
              \usebeamercolor{section title}%
              \usebeamerfont{section title}\insertsubsectionhead\par
          \end{FlushRight}
        \end{minipage}
    };%
  \end{tikzpicture}
}

% ---------------------------------------------------------------------------
%
% Note page (default) updated!
%
\defbeamertemplate{note page}{mydefault}
{%
  {%
    \scriptsize
    \usebeamerfont{note title}\usebeamercolor[fg]{note title}%
    \ifbeamercolorempty[bg]{note title}{}{%
      \insertvrule{.25\paperheight}{note title.bg}%
      \vskip-.25\paperheight%
      \nointerlineskip%
    }%
    \vbox{%
      \hspace*{.6875\paperwidth}\insertslideintonotes{0.25}%
      \vskip-0.25\paperheight%
      \nointerlineskip
      \begin{pgfpicture}{0cm}{0cm}{0cm}{0cm}
        \begin{pgflowlevelscope}{\pgftransformrotate{90}}
          {\pgftransformshift{\pgfpoint{-2cm}{0.2cm}}%
          \pgftext[base,left]{\usebeamerfont{note date}\usebeamercolor[fg]{note date}\the\year-\ifnum\month<10\relax0\fi\the\month-\ifnum\day<10\relax0\fi\the\day}}
        \end{pgflowlevelscope}
      \end{pgfpicture}
    }%
    \nointerlineskip
    \vbox to .25\paperheight{\vskip0.5em
      \hbox{\insertshorttitle[width=8cm]}%
      \setbox\beamer@tempbox=\hbox{\insertsection}%
      \hbox{\ifdim\wd\beamer@tempbox>1pt{\hskip4pt\raise3pt\hbox{\vrule
            width0.4pt height7pt\vrule width 9pt
            height0.4pt}}\hskip1pt\hbox{\begin{minipage}[t]{7.5cm}\def\breakhere{}\insertsection\end{minipage}}\fi%
      }%
      \setbox\beamer@tempbox=\hbox{\insertsubsection}%
      \hbox{\ifdim\wd\beamer@tempbox>1pt{\hskip17.4pt\raise3pt\hbox{\vrule
            width0.4pt height7pt\vrule width 9pt
            height0.4pt}}\hskip1pt\hbox{\begin{minipage}[t]{7.5cm}\def\breakhere{}\insertsubsection\end{minipage}}\fi%
      }%
      \setbox\beamer@tempbox=\hbox{\insertshortframetitle}%
      \hbox{\ifdim\wd\beamer@tempbox>1pt{\hskip30.8pt\raise3pt\hbox{\vrule
            width0.4pt height7pt\vrule width 9pt
            height0.4pt}}\hskip1pt\hbox{\insertshortframetitle[width=7cm]}\fi%
      }%
      \vfil}%
  }%
  \ifbeamercolorempty[bg]{note page}{}{%
    \nointerlineskip%
    \insertvrule{.75\paperheight}{note page.bg}%
    \vskip-.75\paperheight%
  }%
  \vskip.25em
  \nointerlineskip
  %
  %\textwidth=10.8cm%
  \textwidth=13.8cm%
  \hsize=\textwidth%
  %
  \insertnote
}
%
% ---------------------------------------------------------------------------
% Minimal Setup
%
%\setbeamertemplate{subsection page}[minimal]
%\setbeamertemplate{note page}[mydefault]
%\setbeamertemplate{title page}[minimal]
%\setbeamertemplate{headline}[minimal]
%\setbeamertemplate{footline}[nofootline]
%\setbeamertemplate{frametitle}[minimal]
%\setbeamertemplate{section page}[minimal]
%
% ---------------------------------------------------------------------------
% Normal Setup
%
\setbeamertemplate{subsection page}[normal]
\setbeamertemplate{note page}[mydefault]
\setbeamertemplate{title page}[normal]
\setbeamertemplate{headline}[sectioninhead]
\setbeamertemplate{footline}[normal]
\setbeamertemplate{frametitle}[normal]
\setbeamertemplate{section page}[normal]
%
% ===========================================================================
%
