\documentclass[8pt,landscape]{scrartcl}
\usepackage{lmodern}
\usepackage{amssymb,amsmath}
\usepackage{ifxetex,ifluatex}
\usepackage{fixltx2e} % provides \textsubscript
\ifnum 0\ifxetex 1\fi\ifluatex 1\fi=0 % if pdftex
  \usepackage[T1]{fontenc}
  \usepackage[utf8]{inputenc}
\else % if luatex or xelatex
  \ifxetex
    \usepackage{mathspec}
  \else
    \usepackage{fontspec}
  \fi
  \defaultfontfeatures{Ligatures=TeX,Scale=MatchLowercase}
\fi
% use upquote if available, for straight quotes in verbatim environments
\IfFileExists{upquote.sty}{\usepackage{upquote}}{}
% use microtype if available
\IfFileExists{microtype.sty}{%
\usepackage{microtype}
\UseMicrotypeSet[protrusion]{basicmath} % disable protrusion for tt fonts
}{}
\usepackage[left=1cm,right=1cm,top=1cm,bottom=1cm,landscape]{geometry}
\usepackage{hyperref}
\hypersetup{unicode=true,
            pdfborder={0 0 0},
            breaklinks=true}
\urlstyle{same}  % don't use monospace font for urls
\IfFileExists{parskip.sty}{%
\usepackage{parskip}
}{% else
\setlength{\parindent}{0pt}
\setlength{\parskip}{6pt plus 2pt minus 1pt}
}
\setlength{\emergencystretch}{3em}  % prevent overfull lines
\providecommand{\tightlist}{%
  \setlength{\itemsep}{0pt}\setlength{\parskip}{0pt}}
\setcounter{secnumdepth}{0}
% Redefines (sub)paragraphs to behave more like sections
\ifx\paragraph\undefined\else
\let\oldparagraph\paragraph
\renewcommand{\paragraph}[1]{\oldparagraph{#1}\mbox{}}
\fi
\ifx\subparagraph\undefined\else
\let\oldsubparagraph\subparagraph
\renewcommand{\subparagraph}[1]{\oldsubparagraph{#1}\mbox{}}
\fi
%
% ZusatzPakete
% ============
%
%  Die folgende Pakete werden noch benötigt:
%
% ---------------------------------------------------------------------------
%\usepackage{systeme}
\usepackage{booktabs}
\usepackage{xspace}
\usepackage{mathtools}
\usepackage{array}
\usepackage[utf8]{inputenc}
\usepackage[ngerman]{babel}
\usepackage{multicol}
\usepackage{amsmath}
\usepackage{amsfonts}
\usepackage{amssymb}
\usepackage{gensymb}
\usepackage{dsfont}
\usepackage{calc}
\usepackage{csquotes}
\usepackage{enumitem}
\usepackage{xcolor}
\definecolor{FOMVoll}{RGB}{0,153,138}
\newcommand{\mydesrcitem}[1]{\textcolor{FOMVoll!70}{#1}}
\setlist[description]{style=nextline,font=\mydesrcitem}
%\usepackage[permil]{overpic} \usepackage{graphicx} \graphicspath{{gfx/}
% ---------------------------------------------------------------------------------------
%
% DeclareOwnMathOperators
% =======================
%
%  Ein paar eigene mathematische Operatoren definieren
%
% ---------------------------------------------------------------------------
\DeclareMathOperator*{\leftlim}{\text{l-\!}\lim} 		% l-lim
\DeclareMathOperator*{\rightlim}{\text{r-\!}\lim} 	% r-lim
\DeclareMathOperator*{\grad}{\text{grad}} 			% grad
% ---------------------------------------------------------------------------------------
\DeclareTextFontCommand{\textbf}{\bfseries\color{red}}
\DeclareTextFontCommand{\textit}{\bfseries\color{blue}}



\date{}

\begin{document}

% ---------------------------------------------------------------------------
%
% BeforeBodyInclude
% =================
%
%
% ---------------------------------------------------------------------------


\setlength{\columnsep}{1cm}
\begin{multicols}{3}
% ---------------------------------------------------------------------------

\subsection{Logik}\label{logik}

\begin{description}
\tightlist
\item[Aussagen]
Sätze, die entweder \emph{wahr} oder \emph{falsch} sind, heißen
\textbf{Ausagen}.
\item[Aussageformen / offene Aussagen]
Hängte die Wahrheit einer Aussage von einem Parameter \(x\) ab, so nennt
man die Aussage \(A(x)\) eine **offene Aussage* oder *Aussageform**.
\item[Lösungsmenge]
Die Menge der Werte \(x\), die eine Aussageform \(A(x)\) zu einer
\emph{wahren Aussage} machen heißt \textbf{Lösungemenge}
\end{description}

Es seien \(A\) und \(B\) Aussagen, dann gilt:

\begin{description}
\tightlist
\item[Implikation \emph{(Aus \(A\) folge \(B\))}]
\(A \Longrightarrow B\) : falls \(A\) wahr ist, dann ist auch \(B\)
wahr.
\item[Äquivalenz]
\(A \Longleftrightarrow B\) : \(A\) ist genau dann wahr, falls \(B\)
wahr ist.
\item[Konjunktion]
\(A \wedge B\) : \(A\) ist wahr und \(B\) ist wahr.
\item[Disjunktion]
\(A \vee B\) : \(A\) ist wahr oder \(B\) ist wahr.
\item[Negation]
\(-A\) ist wahr \(\Longleftrightarrow\) \(A\) ist falsch.
\item[Allquantor]
\(\forall\) : \enquote{Für alle}"
\item[Existenzquantor]
\(\exists\) : \enquote{Es gibt ein}
\end{description}

\subsection{Mengenlehre}\label{mengenlehre}

Für beliebige Mengen \(A\) und \(B\) gilt:

\begin{description}
\tightlist
\item[Element]
Ist \(a\) ist ein Element von \(A\), dann schreiben wir \(a \in A\).
\item[Teilmenge]
\(A \subset B \Longleftrightarrow \left(x \in A \Rightarrow x \in B\right)\)
\item[Echte Teilmenge]
\(A \subsetneq B \Longleftrightarrow \left(A \subset B \wedge \exists z \in B : z \notin A\right)\)
\item[Gleichheit von Mengen]
\(A = B \Longleftrightarrow A \subset B \wedge B \subset A\)
\item[Vereinigungsmenge zweier Mengen]
\(A \cup B = \{ x | x \in A \vee x \in B \}\)
\item[Schnittmenge zweier Mengen]
\(A \cap B = \{ x | x \in A \wedge x \in B \}\)
\item[Kompliment einer Menge]
\(A^c = \{x | x \in U \wedge x \not\in A \}, U\) ein Universum mit
\(A \subset U\)
\item[Differenz von Mengen]
\(A \setminus B = \{ x | x \in A \wedge x \notin B \} = A \cap B^c\)
\item[Potenzmenge]
\(\mathcal{P}(A) = \{ U | U \subset A\}\)
\item[Disjunktheit]
\(A\) und \(B\) disjunkt \(\Longleftrightarrow A \cap B = \emptyset\)
\item[Gleichmächtigkeit von Mengen]
\(A\) und \(B\) sind gleichmächtig, falls es eine Bijektion
\(f: A \leftrightarrow B\) gibt.
\item[Endlichkeit]
Eine Menge ist \textbf{endlich}, wenn sie \emph{gleichmächtig} zu einem
Element von \(\mathbb{N}_0\) im Sinne von von Neumann ist.
\item[Abzählbar]
Eine Menge ist \textbf{abzählbar}, wenn sie \emph{endlich} ist oder
\emph{gleichmächtig} zu einer \emph{Teilmenge} von \(\mathbb{N}\) ist.
\item[Unendlichkeit]
Eine nicht \emph{endliche} Menge ist \textbf{unendlich}
\item[Mächtigkeit von Mengen (allgemein)]
\(|A|\) heißt \emph{Betrag} der Menge \(A\) und bezeichnet die
Mächtigkeit der Menge.
\item[Mächtigkeit von endlichen Mengen]
\(|A|\) ist die Anzahl der unterscheidbaren Elemente der (endlichen)
Menge \(A\).
\item[Produktmenge]
\(A \times B = \{(x;y) | x \in A \wedge y \in B\}\)
\item[De Morgansche Regeln]
\((A \cup B)^c = A^c \cap B^c\) und \((A \cap B)^c = A^c \cup B^c\)
\item[Disjunktheit]
\(A\) und \(B\) sind \emph{disjunkt}
\(\Longleftrightarrow A \cap B = \emptyset\)
\item[Zerlegung / Partition]
Die Mengen \(A_1, ..., A_n\) mit
\(A_1 \cup A_2 \cup \cdots \cup A_n = A\) und
\(A_i \cap A_j = \emptyset\) für alle \(0 \leq i \not= j \leq n\) heißt
\emph{Partition} oder \emph{Zerlegung} von \(A\).
\end{description}

\subsection{Zahlen}\label{zahlen}

\begin{description}
\tightlist
\item[Natürliche Zahlen]
\(\displaystyle \mathbb{N} = \{1,2,3,4,...\}\)
\item[Natürliche Zahlen mit Null:]
\(\displaystyle \mathbb{N}_0 = \mathbb{N} \cup \{0\} = \{0,1,2,3,4,...\}\)
\item[Ganze Zahlen]
\(\displaystyle \mathbb{Z} = \{...,-3,-2,-1,0,1,2,3,...\}\)
\item[Rationale Zahlen]
\(\displaystyle \mathbb{Q} = \left\{ \frac{q}{p} \left| q\in \mathbb{Z}, p \in \mathbb{N}, p \text{ und } q \text{ sind teilerfremd }\right.\right\}\)
\item[Reelle Zahlen]
\(\displaystyle \mathbb{R}\)
\item[Komplexe Zahlen]
\(\displaystyle \mathbb{C} = \left\{ x+y\cdot i \left| x,y \in \mathbb{R}\right. \right\}\)
\end{description}

Es gilt:

\begin{equation*}
    \mathbb{N} \subsetneq  \mathbb{N}_0 \subsetneq  \mathbb{Z} \subsetneq \mathbb{Q} \subsetneq \mathbb{R}\subsetneq \mathbb{C}
\end{equation*}

\subsection{Folgen und Reihen}\label{folgen-und-reihen}

Rechenregeln für konvergente Folgen:

Seien \((a_n)\) und \((b_n)\) konvergente Folgen mit den Grenzwerten
\(a\) und \(b\). Weiter sei \(c \in \mathbb{R}\). Dann gilt:

\begin{itemize}
\tightlist
\item
  \(\lim\limits_{n \to \infty} (c \cdot a_n) = c \cdot a\)
\item
  \(\lim\limits_{n \to \infty} (a_n \pm b_n) = a \pm b\)
\item
  \(\lim\limits_{n \to \infty} (a_n \cdot b_n) = a \cdot b\)
\item
  \(\lim\limits_{n \to \infty} \left(\frac{a_n}{b_n}\right) = \frac{a}{b},\)
  falls \(b \not= 0\)
\end{itemize}

\subsection{Kombinatorik}\label{kombinatorik}

\begin{description}
\tightlist
\item[Summenregel]
\(|A \cup B| = |A| + |B| - |A \cap B|\)
\item[Produktregel]
\(|A \times B| = |A| \cdot |B|\)
\item[Binomischer Lehrsatz]
Für zwei reelle Zahlen \(a\), \(b\) und eine natürliche Zahl \(n\) gilt:

\((a+b)^n = \sum\limits_{k=0}^{n} \binom{k}{n} a^{n-k}b^{k}\)
\end{description}

Für die Anzahl der Möglichkeiten aus \(n\) Objekten \(k\) Objekte
auszuwählen, gelten die folgenden Regeln:

\begin{center}
    \begin{tabular}{lcc}
        \toprule
        Auswahl & \textbf{mit} Beachtung    & \textbf{ohne} Beachtung   \\
        ~       & der Reihenfolge           & der Reihenfolge           \\
        ~       & (\textit{Variation})      & (\textit{Kombination})    \\
        \midrule
        ohne Zurücklegen & $\displaystyle\frac{n!}{(n-k)!}$ & $\displaystyle\binom{n}{k}$ \\
        \midrule
        mit  Zurücklegen & $\displaystyle n^k$              & $\displaystyle\binom{n+k-1}{k}$ \\
        \bottomrule
    \end{tabular}
\end{center}

\subsection{Wahrscheinlichkeitsrechnung}\label{wahrscheinlichkeitsrechnung}

In einem Wahrscheinlichkeitsraum \((\Omega, \Sigma, P)\) ist \(\Omega\)
die Ergebnismenge, \(\Sigma\) der Ereignisraum und \(P\) ein
Wahscheinlichkeitsmaß.

Es gilt dann für die beliebigen Ereignisse \(A\) und \(B\) bzw. die
disjunkten Ereignisse \(A_1,...,A_n\) aus \(\Sigma\):

\begin{description}
\item[Gegenereignis von Ereignis \(A\)]
\(\overline{A} = A^c = \Omega \setminus A\)
\item[Sicheres Ereignis]
\(\Omega\)
\item[Unmögliches Ereignis]
\(\emptyset\) oder \(\{\}\)
\item[Teilereignis \(A\) von \(B\)]
\(A \subset B\)
\item[Disjunktheit / Unverträglichkeit]
\(A\) und \(B\) sind \emph{disjunkt} oder \emph{unverträglich}
\(\Longleftrightarrow A \cap B = \emptyset\)
\item[Nichtnegativität der Wahrscheinlichkeitsfunktion]
\(P(A) \in [0; 1]\)
\item[Normiertheit der Wahrscheinlichkeitsfunktion]
\(P(\Omega) = 1\)
\item[Gegenwahrscheinlichkeit]
\(P(\overline{A})=1-P(A)\)
\item[Wahrscheinlichkeit des unmöglichen Ereignisses]
\(P(\emptyset)=0\)
\item[Summerregel]
\(P(A \cup B) = P(A) + P(B) - P(A \cap B)\)
\item[Additivität]
Für eine paarweise disjunkte Ereignisse \(A_1,...,A_n\) gilt:

\(P(A_1 \cup A_2 \cup ... \cup A_n) = P(A_1) + P(A_2) + \cdots + P(A_n)\)
\item[Stochastische Unabhänigkeit]
\(A\) und \(B\) sind unabhängig
\(\Longleftrightarrow P(A \cap B) = P(A) \cdot P(B)\)
\item[Bedingte Wahrscheinlichkeit von \(A\) unter der Bedingung \(B\)]
\begin{equation*}
    P(A | B) = \frac{P(A \cap B}{P(B)}
\end{equation*}
\item[Multiplikationssatz]
\(P(A\cap B) = P(A\mid B) \cdot P(B)\)
\item[Satz von der totalen Wahrscheinlichkeit]
Sei \(\{E_1, \dots, E_k\}\) eine \emph{Zerlegung} von \(\Omega\) mit
\(P(E_i) > 0\). Dann ist

\begin{equation*}
    P(E) = \sum\limits_{i=1}^{k} P(E_i) \cdot P(E \mid E_i)
\end{equation*}
\item[Satz von Bayes]
Für zwei Ereignisse \(A\) und \(B\) mit \(P(B) > 0\) gilt:

\begin{equation*}
    P(A \mid B) \; = \; \frac {P(B \mid A) \cdot P(A)} {P(B)}.
\end{equation*}
\item[Satz von Bayes für Gegenereignisse]
Da ein Ereignis \(A\) und sein Gegenereignis \(\bar{A}\) stets eine
Zerlegung der

Ergebnismenge darstellen, gilt insbesondere:

\begin{equation*} 
    P(A \mid B) \; = \; \frac{P(B \mid A) \cdot P(A)}{P(B \mid A) \cdot P(A) + P(B \mid \bar{A}) \cdot P(\bar{A})}.
\end{equation*}
\item[Klassische Wahrscheinlichkeitsfunktion bei Laplace-Experimenten]
\begin{equation*}
    P(A) = \frac{\text{\enquote{Anzahl der für A günstigen Fälle}}}{\text{\enquote{Anzahl der möglichen Fälle}}}
\end{equation*}
\end{description}

%
% AfterBodyInclude
% ================
%
%
% ---------------------------------------------------------------------------
\end{multicols}
% ---------------------------------------------------------------------------------------

\end{document}
