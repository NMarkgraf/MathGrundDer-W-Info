\PassOptionsToPackage{unicode=true}{hyperref} % options for packages loaded elsewhere
\PassOptionsToPackage{hyphens}{url}
%
\documentclass[9pt,ngerman,a4paper,landscape]{scrartcl}
\usepackage{lmodern}
\usepackage{amssymb,amsmath}
\usepackage{ifxetex,ifluatex}
\usepackage{fixltx2e} % provides \textsubscript
\ifnum 0\ifxetex 1\fi\ifluatex 1\fi=0 % if pdftex
  \usepackage[T1]{fontenc}
  \usepackage[utf8]{inputenc}
  \usepackage{textcomp} % provides euro and other symbols
\else % if luatex or xelatex
  \usepackage{unicode-math}
  \defaultfontfeatures{Ligatures=TeX,Scale=MatchLowercase}
\fi
% use upquote if available, for straight quotes in verbatim environments
\IfFileExists{upquote.sty}{\usepackage{upquote}}{}
% use microtype if available
\IfFileExists{microtype.sty}{%
\usepackage[]{microtype}
\UseMicrotypeSet[protrusion]{basicmath} % disable protrusion for tt fonts
}{}
\IfFileExists{parskip.sty}{%
\usepackage{parskip}
}{% else
\setlength{\parindent}{0pt}
\setlength{\parskip}{6pt plus 2pt minus 1pt}
}
\usepackage{hyperref}
\hypersetup{
            pdfborder={0 0 0},
            breaklinks=true}
\urlstyle{same}  % don't use monospace font for urls
\usepackage[left=1cm,right=1cm,top=1cm,bottom=1cm,landscape]{geometry}
\setlength{\emergencystretch}{3em}  % prevent overfull lines
\providecommand{\tightlist}{%
  \setlength{\itemsep}{0pt}\setlength{\parskip}{0pt}}
\setcounter{secnumdepth}{0}
% Redefines (sub)paragraphs to behave more like sections
\ifx\paragraph\undefined\else
\let\oldparagraph\paragraph
\renewcommand{\paragraph}[1]{\oldparagraph{#1}\mbox{}}
\fi
\ifx\subparagraph\undefined\else
\let\oldsubparagraph\subparagraph
\renewcommand{\subparagraph}[1]{\oldsubparagraph{#1}\mbox{}}
\fi

% set default figure placement to htbp
\makeatletter
\def\fps@figure{htbp}
\makeatother

%
% ZusatzPakete
% ============
%
%  Die folgende Pakete werden noch benötigt:
%
% ---------------------------------------------------------------------------
%\usepackage{systeme}
\usepackage{booktabs}
\usepackage{xspace}
\usepackage{mathtools}
\usepackage{array}
\usepackage[utf8]{inputenc}
%\usepackage[ngerman]{babel}
\usepackage{multicol}
\usepackage{amsmath}
\usepackage{amsfonts}
\usepackage{amssymb}
\usepackage{gensymb}
\usepackage{dsfont}
\usepackage{calc}
\usepackage{csquotes}

\usepackage{enumitem}
\usepackage{xcolor}
\definecolor{FOMVoll}{RGB}{0,153,138}
\newcommand{\mydesrcitem}[1]{\textcolor{FOMVoll!90}{#1}}
%\setlist[description]{style=nextline,font=\sffamily\mydesrcitem}
\setlist[description]{style=nextline,font=\normalfont\mydesrcitem,itemsep=0.8em}

\DeclareTextFontCommand{\textbf}{\bfseries\color{red}}
\DeclareTextFontCommand{\emph}{\bfseries\color{blue}}

%\usepackage[permil]{overpic} \usepackage{graphicx} \graphicspath{{gfx/}
% ---------------------------------------------------------------------------------------
%
% DeclareOwnMathOperators
% =======================
%
%  Ein paar eigene mathematische Operatoren definieren
%
% ---------------------------------------------------------------------------
\DeclareMathOperator*{\leftlim}{\text{l-\!}\lim} 		% l-lim
\DeclareMathOperator*{\rightlim}{\text{r-\!}\lim} 	% r-lim
\DeclareMathOperator*{\grad}{\text{grad}} 			% grad
\DeclareMathOperator*{\Pot}{\mathcal{P}}
% ---------------------------------------------------------------------------------------
\ifnum 0\ifxetex 1\fi\ifluatex 1\fi=0 % if pdftex
  \usepackage[shorthands=off,main=ngerman]{babel}
\else
  % load polyglossia as late as possible as it *could* call bidi if RTL lang (e.g. Hebrew or Arabic)
  \usepackage{polyglossia}
  \setmainlanguage[]{german}
\fi

\date{}

\begin{document}

% ---------------------------------------------------------------------------
%
% BeforeBodyInclude
% =================
%
%
% ---------------------------------------------------------------------------

%{\Huge\sffamily\textcolor{FOMVoll}{{\bfseries Formelsammlung} zur Klausur \enquote{Mathematik für Wirtschaftsinformatiker}}}
{\Huge\sffamily\textcolor{FOMVoll}{{\bfseries Formelsammlung} zur Klausur \enquote{Mathematische Grundlagen der Wirtschaftsinformatik}}}

\setlength{\columnsep}{0.75cm}
\begin{multicols}{3}
% ---------------------------------------------------------------------------

\hypertarget{notationen}{%
\subsection{Notationen}\label{notationen}}

\begin{description}
\tightlist
\item[Summenzeichen]
\(\displaystyle \sum\limits_{k=m}^n a_k = a_m + a_{m+1} + a_{m+2} + \dotsc + a_{n-1} + a_{n}\)
\item[Produktzeichen]
\(\displaystyle \prod\limits_{k=m}^n a_k = a_m \cdot a_{m+1} \cdot a_{m+2} \cdot \dotsc \cdot a_{n-1} \cdot a_{n}\)
\item[Fakultät]
\(\displaystyle n! = \prod\limits_{k=1}^n k = 1 \cdot 2 \cdot \dotsc \cdot (n-1) \cdot n\)

\(\displaystyle 0! = 1\)
\end{description}

\hypertarget{einfaches-rechnen}{%
\subsection{Einfaches Rechnen}\label{einfaches-rechnen}}

\begin{description}
\item[Betrag]
Für eine reelle Zahl \(x\) ist der \textbf{(Absolut-)Betrag} definiert
durch:

\(\displaystyle |x| = \sqrt{x^2} = \begin{cases} x : x > 0 \\ 0 : x=0 \\ -x : x <0 \end{cases}\)
\item[Rechnen mit Beträgen]
Für reelle Zahlen \(x\),\(y\) und eine nicht-negative reelle Zahl \(p\)
gelten die folgenden Regeln:

\begin{tabular}{ l c l }
    $|x| \geq 0$                                    & $\qquad$  &   $|x| = 0 \Longleftrightarrow x=0$ \\
    $|x \cdot y| = |x| \cdot |y|$                   &   & \\
    $|x \cdot p| = |x| \cdot p$                     & $\qquad$  &   $|x \cdot (-p)| = |x| \cdot p$ \\
    $|x+y| \leq |x| + |y|$                          & $\qquad$  &   $|x-y| \geq \left| |x| - |y| \right|$ \\
    $\left|\frac{x}{y}\right| = \frac{|x|}{|y|}$    & $\qquad$  &   \\
\end{tabular}
\item[Bruchrechnen]
Für alle Zahlen \(a\), \(b\), \(c\), \(d\) mit \(c\neq 0\) und
\(d \neq 0\) gilt:

\begin{tabular}{ l c l }
    $\displaystyle \frac{a}{c} + \frac{b}{d} = \frac{ad+bc}{cd}$    & $\qquad$  &   $\displaystyle \frac{a}{c} - \frac{b}{d} = \frac{ad-bc}{cd}$ \\
    $\displaystyle \frac{c\cdot a}{c \cdot d} = \frac{a}{d}$        & $\qquad$  &   $\displaystyle \frac{a}{c} \cdot \frac{b}{d} = \frac{ab}{cd}$ \\
    $\displaystyle \frac{\frac{a}{c}}{\frac{b}{d}} = \frac{ad}{bc}$ & & \\
\end{tabular}
\item[Potenzrechengesetze]
Für reelle Zahlen \(a\neq0\) und \(b\neq0\), reelle Zahlen \(r\) und
\(s\) falls \(a>0\) und rationale Zahlen \(r\) und \(s\) falls \(a<0\)
ist gilt:

\begin{tabular}{ l c l }
    $\displaystyle a^0 = 1$                 & $\qquad$  & $\displaystyle a^{-r} = \frac{1}{a^r}$ \\
    $\displaystyle a^{r+s} = a^r \cdot a^s$ & $\qquad$  & $\displaystyle a^{r-s} = \frac{a^r}{a^s}$ \\
    $\displaystyle (a \cdot b)^{r} = a^r \cdot b^r$ & $\qquad$  & $\displaystyle \left(\frac{a}{b}\right)^{r} = \frac{a^r}{b^r}$ \\
    $\displaystyle (a^r)^{s} = a^{r \cdot s}$ & & \\
\end{tabular}

Für positive Zahlen \(a\) kann man die Potenz durch Exponentialfunktion
und Logaritmus ausdrücken:

\(\displaystyle x^{r} = \exp\left(r \cdot \ln(x)\right)\)
\item[Wurzelrechnengesetze]
Für positive Zahlen \(a\) und \(b\) und \(n,m,k \in \mathbb{N}\) gilt:

\begin{tabular}{ l c l }
$\displaystyle \sqrt[n]{a}\cdot\sqrt[n]{b}=\sqrt[n]{a\cdot b}$ & & 
$\displaystyle \frac{\sqrt[n]{a}}{\sqrt[n]{b}}=\sqrt[n]{\frac{a}{b}}$ \\
$\displaystyle \sqrt[k]{\sqrt[n]{a}}=\sqrt[k\cdot n]{a}$ & & 
$\displaystyle a^{\frac{m}{n}}=\sqrt[n]{a^m}=\left(\sqrt[n]{a} \right)^m$ \\
$\displaystyle a^{-\frac{m}{n}}=\frac{1}{a^\frac{m}{n}}$ & &
$\displaystyle \sqrt[n]{a}\cdot\sqrt[m]{a}=a^{\frac{1}{n}+\frac{1}{m}}=\sqrt[nm]{a^{n+m}}$ \\
\end{tabular}

Höhere Wurzeln aus positiven Zahlen \(x\) kann man wie jede Potenz durch
Exponentialfunktion und Logarithmus ausdrücken:

\(\displaystyle \sqrt[n]{x} = x^{1/n} = \exp\left(\frac{\ln(x)}{n}\right)\)
\item[Logarithmengesetze]
Für reellen, positive Zahlen \(a,b, x, y\) mit \(a, b \neq 1\), einem
reellen \(r\) und einer natürlichen Zahl \(n\) gilt:

\begin{tabular}{ l c l }
    $\displaystyle \log_a(1) = 0$ & & \\
\end{tabular}

\begin{tabular}{ l c l c l}
    $\displaystyle \text{lb}(x) = \log_2(x)$ & $\;$ & $\displaystyle \ln(x) = \log_e(x)$ & $\;$ & $\displaystyle \text{lg}(x) = \log_{10}(x)$ \\
\end{tabular}

\begin{tabular}{ l }
    $\displaystyle \log_a (x \cdot y) = \log_a(x) + \log_a(y)$ \\
    $\displaystyle \log_a \left(\frac{x}{y}\right) = \log_a(x) - \log_a(y)$ \\
    $\displaystyle \log_a(x^r) = r \cdot \log_a(x)$ \\
    $\displaystyle \log_a\left(\frac{1}{x}\right) = - \log_a(x)$ \\
    $\displaystyle \log_a(x + y) = \log_a(x) + \log_a\left(1+ \frac{x}{y}\right)$ \\
    $\displaystyle \log_b\left(\sqrt[n]{x}\right) = \log_b \left(x^{\frac 1n}\right) = \frac 1n\log_b x$ \\
    $\displaystyle \log_a(x) = \frac{\log_b(x)}{\log_b(a)}$
\end{tabular}
\item[Binomische Formeln]
Für reelle Zahlen \(x\) und \(y\) gelten die folgenden Regeln:

\(\displaystyle (x+y)^2 = x^2 + 2 xy + y^2\)

\(\displaystyle (x-y)^2 = x^2 - 2 xy + y^2\)

\(\displaystyle (x-y)(x+y) = x^2-y^2\)
\item[Binomischer Lehrsatz]
Für zwei reelle Zahlen \(x\), \(y\) und eine natürliche Zahl \(n\) gilt:

\(\displaystyle (x+y)^n = \sum\limits_{k=0}^{n} \dbinom{k}{n} x^{n-k}y^{k}\)
\item[Normalform von Polynomgleichungen]
Jede Polynomgleichung (2. Grades) der Form \(a x^2+ bx +c = d\), mit
\(a\neq 0\) lässt sich umformen in \textbf{Normalform} der Art
\(x^2+px+q=0\).
\item[Diskriminante]
Für eine Polynomgleichung (2. Grades) ist die \textbf{Diskriminante}
definiert durch \(D=\frac{p^2-4\cdot q}{4}\).

Es gilt:

\begin{itemize}
\tightlist
\item
  \(D < 0\): die Gleichung hat keine (reelle) Lösung!
\end{itemize}

\begin{itemize}
\tightlist
\item
  \(D = 0\): die Gleichung hat eine Lösung nämlich \(-\frac{p}{2}\).
\end{itemize}

\begin{itemize}
\tightlist
\item
  \(D > 0\): die Gleichung hat zwei Lösungen. (-\textgreater{}
  pq-Formel)
\end{itemize}
\item[pq-Formel]
Für eine Polynomgleichung (2. Grades) mit positiver Diskriminante findet
sich die Nullstellen \(x_{1/2}\) durch

\(\displaystyle x_{1/2} = -\frac{p}{2} \pm \sqrt{D} = -\frac{p}{2} \pm \sqrt{\left(\frac{p}{2}\right)^2 - q}\)
\item[Satz von Vieta]
Für die Lösungen \(x_1\) und \(x_2\) einer Polynomgleichung (2. Grades)
in Normalform gilt:

\(x_1 \cdot x_2 = q\) und \(-(x_1+x_2)=p\)
\end{description}

\hypertarget{mengenlehre}{%
\subsection{Mengenlehre}\label{mengenlehre}}

Für beliebige Mengen \(A\) und \(B\) gilt:

\begin{description}
\tightlist
\item[Element]
Ist \(a\) ist ein \textbf{Element} von \(A\), dann schreiben wir
\(a \in A\).
\item[Teilmenge]
\(\displaystyle A \subset B \Longleftrightarrow \left(x \in A \Rightarrow x \in B\right)\)
\item[Echte Teilmenge]
\(\displaystyle A \subsetneq B \Longleftrightarrow \left(A \subset B \wedge \exists z \in B : z \notin A\right)\)
\item[Gleichheit von Mengen]
\(\displaystyle A = B \Longleftrightarrow A \subset B \wedge B \subset A\)
\item[Vereinigungsmenge zweier Mengen]
\(\displaystyle A \cup B = \{ x | x \in A \vee x \in B \}\)
\item[Schnittmenge zweier Mengen]
\(\displaystyle A \cap B = \{ x | x \in A \wedge x \in B \}\)
\item[Kompliment einer Menge]
\(\displaystyle A^c = \{x | x \in U \wedge x \not\in A \}, U\) ein
Universum mit \(A \subset U\)
\item[Differenz von Mengen]
\(\displaystyle A \setminus B = \{ x | x \in A \wedge x \notin B \} = A \cap B^c\)
\item[Gleichmächtigkeit von Mengen]
\(A\) und \(B\) sind gleichmächtig, falls es eine Bijektion
\(f: A \leftrightarrow B\) gibt.
\item[Endlichkeit]
Eine Menge ist \textbf{endlich}, wenn sie \emph{gleichmächtig} zu einem
Element von \(\mathbb{N}_0\) im Sinne von von Neumann ist.
\item[Abzählbar]
Eine Menge ist \textbf{abzählbar}, wenn sie \emph{endlich} ist oder
\emph{gleichmächtig} zu einer \emph{Teilmenge} von \(\mathbb{N}\) ist.
\item[Unendlichkeit]
Eine nicht \emph{endliche} Menge ist \textbf{unendlich}
\item[Mächtigkeit von Mengen (allgemein)]
\(|A|\) heißt \emph{Betrag} der Menge \(A\) und bezeichnet die
Mächtigkeit der Menge.
\item[Mächtigkeit von endlichen Mengen]
\(|A|\) ist die Anzahl der unterscheidbaren Elemente der (endlichen)
Menge \(A\).
\item[Potenzmenge]
\(\displaystyle \Pot(A) = \{ U | U \subset A\}\)
\item[Satz von Cantor]
Für jede Menge \(A\) gilt: \(|A| < |\Pot(A)|\)
\item[Produktmenge]
\(\displaystyle A \times B = \{(x;y) | x \in A \wedge y \in B\}\)
\item[De Morgansche Regeln]
\(\displaystyle (A \cup B)^c = A^c \cap B^c\) und
\(\displaystyle (A \cap B)^c = A^c \cup B^c\)
\item[Disjunktheit]
\(A\) und \(B\) sind \emph{disjunkt}
\(\displaystyle\Longleftrightarrow A \cap B = \emptyset\)
\item[Zerlegung / Partition]
Die Mengen \(A_1, ..., A_n\) mit
\(A_1 \cup A_2 \cup \cdots \cup A_n = A\) und
\(A_i \cap A_j = \emptyset\) für alle \(0 \leq i \not= j \leq n\) heißt
\emph{Partition} oder \emph{Zerlegung} von \(A\).
\end{description}

\hypertarget{zahlen}{%
\subsection{Zahlen}\label{zahlen}}

\begin{description}
\tightlist
\item[Natürliche Zahlen]
\(\displaystyle \mathbb{N} = \{1,2,3,4,...\}\)
\item[Natürliche Zahlen mit Null:]
\(\displaystyle \mathbb{N}_0 = \mathbb{N} \cup \{0\} = \{0,1,2,3,4,...\}\)
\item[Ganze Zahlen]
\(\displaystyle \mathbb{Z} = \{...,-3,-2,-1,0,1,2,3,...\}\)
\item[Rationale Zahlen]
\(\displaystyle \mathbb{Q} = \left\{ \left. \frac{q}{p} \right| q\in \mathbb{Z}, p \in \mathbb{N}, p \text{ und } q \text{ sind teilerfremd}\right\}\)
\item[Reelle Zahlen]
\(\displaystyle \mathbb{R}\)
\item[Komplexe Zahlen]
\(\displaystyle \mathbb{C} = \left\{ x+y\cdot i \left| x,y \in \mathbb{R}\right. \right\}\)
\end{description}

Es gilt: \begin{equation*}
    \mathbb{N} \subsetneq  \mathbb{N}_0 \subsetneq  \mathbb{Z} \subsetneq \mathbb{Q} \subsetneq \mathbb{R}\subsetneq \mathbb{C}
\end{equation*}

\hypertarget{vollstandige-induktion}{%
\subsection{Vollständige Induktion}\label{vollstandige-induktion}}

Sei \(A(n)\) eine Aussageform, die es für alle \(n \in \mathbb{N}\) zu
beweisen gilt

\begin{itemize}
\tightlist
\item
  \textbf{Induktionsanfang:} \(A(1)\) gilt.
\item
  \textbf{Induktionsschritt:} Unter der Annahme das \(A(n)\) gilt zeigt
  man, dass \(A(n+1)\) gilt.

  \begin{itemize}
  \tightlist
  \item
    \textbf{Induktionsannahme}: Es gelte \(A(n)\).
  \item
    \textbf{Induktionsschluss}: Zu zeigen ist dann, dass \(A(n+1)\)
    gilt.
  \end{itemize}
\end{itemize}

\hypertarget{kombinatorik}{%
\subsection{Kombinatorik}\label{kombinatorik}}

\begin{description}
\tightlist
\item[Summenregel]
\(\displaystyle |A \cup B| = |A| + |B| - |A \cap B|\)
\item[Inklusion und Exklusion]
\(\displaystyle |A \cup B \cup C| = |A| + |B| + |C| - |A \cap B| - |A \cap C| - |B \cap C| + |A \cap B \cap C|\)
\item[Produktregel]
\(\displaystyle |A \times B| = |A| \cdot |B|\)
\item[k-Permutationen / Variation]
\(\displaystyle P(n, k) = n \cdot (n-1) \cdot (n-2) \cdot \dotsc \cdot (n-k+1) = \frac{n!}{(n-k)!}\)
\item[Permutation]
\(\displaystyle P(n, n) = n! = n \cdot (n-1) \cdot (n-2) \cdot \dotsc \cdot 1\)
\item[Binomialkoeffizient]
\(\displaystyle \dbinom{n}{k} = C(n,k) = \frac{P(n,k)}{k!} = \frac{n!}{k! \cdot (n-k)!}\)
\end{description}

Für die Anzahl der Möglichkeiten aus \(n\) Objekten \(k\) Objekte
auszuwählen, gelten die folgenden Regeln:

\begin{center}
    \begin{tabular}{lcc}
        \toprule
        Auswahl & \textbf{mit} Beachtung    & \textbf{ohne} Beachtung   \\
        ~       & der Reihenfolge           & der Reihenfolge           \\
        ~       & (\textit{Variation})      & (\textit{Kombination})    \\
        \midrule
        \textbf{ohne} Zurücklegen & $\displaystyle\frac{n!}{(n-k)!}$    & $\displaystyle\dbinom{n}{k}$ \\
        \midrule
        \textbf{mit}  Zurücklegen & $\displaystyle n^k$             & $\dbinom{n+k-1}{k}$ \\
        \bottomrule
    \end{tabular}
\end{center}

\hypertarget{differentialrechnung}{%
\subsection{Differentialrechnung}\label{differentialrechnung}}

\begin{description}
\tightlist
\item[Differentialquotient erster Ordnung]
Die \textbf{Ableitung} oder der \textbf{Differentialquotient} einer
Funktion \(f\) an der Stelle \(x_0\) ist, falls der Grenzwert existiert

\(\displaystyle f'(x_0) = \frac{\text{d} f}{\text{d} x} (x_0) = \lim\limits_{x \to x_0} \frac{f(x_0)-f(x)}{x_0-x}\)
\item[Differentialquotient zweiter Ordnung]
Die \textbf{2. Ableitung} oder der \textbf{Differentialquotient 2.
Ordnung} einer Funktion \(f\) an der Stelle \(_x0\) ist, falls der
Grenzwert existiert, die Ableitung der 1. Ableitung.
\end{description}

\hypertarget{ableitungsregeln}{%
\subsubsection{Ableitungsregeln:}\label{ableitungsregeln}}

Für differenzierbare, reelle Funktionen \(f\), \(g\), \(z\) und \(n\)
gelten die folgenden Regeln:

\begin{description}
\tightlist
\item[Summenregel]
\(\displaystyle [f \pm g]'(x) = f'(x) \pm g'(x)\)
\item[Produktregel]
\(\displaystyle [f \cdot g]'(x) = f'(x) \cdot g(x) + f(x) \cdot g'(x)\)
\item[Produktregel für eine reelle Konstante \(c\)]
\(\displaystyle [c \cdot f]'(x) = c \cdot f'(x)\)
\item[Quotientenregel]
\(\displaystyle \left[\frac{z(x)}{n(x)}\right]' = \frac{z'(x)\cdot n(x) - z(x) \cdot n'(x)}{\left(n(x)\right)^2}\)
\item[Kettenregel]
\(\displaystyle \left[f\left(g(x)\right)\right]' = f'\left( g(x) \right) \cdot f'(x)\)
\end{description}

\hypertarget{ableitung-elementarer-funktionen}{%
\subsubsection{Ableitung elementarer
Funktionen}\label{ableitung-elementarer-funktionen}}

\begin{tabular}{ l c l }
    $\displaystyle\left[\ln(x)\right]' = \frac{1}{x}$                       & $\qquad$  & $\displaystyle\left[e^x\right]' =     e^x$            \\[0.75em]  
    $\displaystyle\left[\log_a (x)\right]' = \frac{1}{x \cdot \ln(a)}$      & $\qquad$  & $\displaystyle\left[a^x\right]' = a^x \cdot \ln (a)$  \\[0.75em]
    $\displaystyle\left[x^b\right]' = b \cdot x^{b-1}$                      & $\qquad$  & $\displaystyle\left[c\right]'  = 0$                   \\[0.75em]
    $\displaystyle\left[\sin(x)\right]' = \cos(x)$                          & $\qquad$  & $\displaystyle\left[\cos(x)\right]'    = -\sin(x)$    \\[0.5em]
\end{tabular}

\hypertarget{multivariate-differentialrechnung}{%
\subsubsection{Multivariate
Differentialrechnung}\label{multivariate-differentialrechnung}}

\begin{description}
\tightlist
\item[Partielle Ableitungen erster Ordnung]
\(\displaystyle \frac{\partial f(x,y)}{\partial x} = f_x(x,y)\)

\(\displaystyle \frac{\partial f(x,y)}{\partial y} = f_y(x,y)\)
\item[Partielle Ableitungen zweiter Ordnung]
\(\displaystyle \frac{\partial^2 f(x,y)}{\partial x\partial x} = f_{xx}(x,y)\)

\(\displaystyle \frac{\partial^2 f(x,y)}{\partial y\partial x} = f_{xy}(x,y)\)

\(\displaystyle \frac{\partial^2 f(x,y)}{\partial x\partial y} = f_{yx}(x,y)\)

\(\displaystyle \frac{\partial^2 f(x,y)}{\partial y\partial y} = f_{yy}(x,y)\)
\end{description}

\hypertarget{hesse-matrix}{%
\subsubsection{Hesse-Matrix}\label{hesse-matrix}}

Für eine bivariate (zweimal partiell stetig differenzierbare) Funktion
\(f(x,y)\) wird durch

\[ 
  A_{Hess(f(x,y))} =
  \begin{pmatrix}
    f_{xx}(x,y) & f_{xy}(x,y) \\
    f_{yx}(x,y) & f_{yy}(x,y)  
  \end{pmatrix}
\]

die \textbf{Hesse-Matrix} definiert.

\begin{description}
\tightlist
\item[Satz von Schwarz]
Für eine bivariate (zweimal partiell stetig differenzierbare) Funktion
\(f(x,y)\) gilt \(f_{xy}(x,y) = f_{yx}(x,y)\).
\end{description}

\hypertarget{definitheit-symmetrischer-2-times-2-matrizen}{%
\subsubsection{\texorpdfstring{Definitheit symmetrischer
(\(2 \times 2\))
Matrizen}{Definitheit symmetrischer (2 \textbackslash{}times 2) Matrizen}}\label{definitheit-symmetrischer-2-times-2-matrizen}}

Eine \((2\times 2)\) Matrix \(A\) ist \textbf{definit}, falls

\[
  \det A = \det \begin{pmatrix} a & c \\ c & b \end{pmatrix} = a \cdot b - c^2 > 0 \text{ und } a \neq 0
\]

gilt. Eine solche \emph{definite} Matrix ist für \(a>0\) \textbf{positiv
definit} und für \(a<0\) \textbf{negativ definit}.

\hypertarget{extremstellen-bivariater-funktionen}{%
\subsubsection{Extremstellen bivariater
Funktionen}\label{extremstellen-bivariater-funktionen}}

\begin{description}
\tightlist
\item[Kritischer Punkt]
\((x_0,y_0)\) ist ein \emph{kritischer Punkt} von \(f(x,y)\), falls
\(f_x(x_0, y_0) = 0\) und \(f_y(x_0, y_0)=0\) gilt.
\item[Extremstellen / Sattelpunkte]
Ein kritischer Punkt \((x_0,y_0)\) ist ein \textbf{Maximum} von
\(f(x,y)\), falls die \emph{Hesse-Matrix} von \(f(x,y)\) an der Stellte
\((x_0,y_0)\) \emph{negativ definit} ist.

Ist die Hesse-Matrix dort \emph{positiv definit}, dann hat \(f(x,y)\)
dort ein \textbf{Minimum}.

Ist die Hesse-Matrix dort \emph{indefinit}, so handelt es sich um einen
\textbf{Sattelpunkt}.
\end{description}

\hypertarget{integralrechnung}{%
\subsection{Integralrechnung}\label{integralrechnung}}

\begin{description}
\tightlist
\item[Stammfunktion]
Eine Funktion \(F\) heißt \textbf{Stammfunktion} von \(f\), falls gilt:

\(\displaystyle F'(x) = f(x)\)
\item[Unbestimmtest Integral]
Damit gilt für das \textbf{unbestimmte Integral}:

\(\displaystyle \int f(x) \,\textrm{d}x \int F'(x) \,\textrm{d}x F(x) + c\)
\end{description}

\hypertarget{elementare-stammfunktionen}{%
\subsubsection{Elementare
Stammfunktionen:}\label{elementare-stammfunktionen}}

\begin{tabular}{ l c l }
    $\displaystyle\int x^n \,\textrm{d}x$           & $\quad=\quad$ & $\displaystyle\frac{1}{n+1}x^{n+1}+c$ \hfil (für $n\neq-1$) \\[0.75em]
    $\displaystyle\int \frac1x \,\textrm{d}x$       & $\quad=\quad$ & $\displaystyle\begin{cases}\ln(x) + c \phantom{-}\quad \text{für } x > 0\\ \ln(-x) + c \quad \text{für } x < 0 \end{cases}$ \hfil (für $n\neq-1$) \\[0.75em]
    $\displaystyle\int e^x \,\textrm{d}x$           & $\quad=\quad$ & $\displaystyle e^x+c$\\[0.75em]
    $\displaystyle\int a^x \,\textrm{d}x$           & $\quad=\quad$ & $\displaystyle \frac{a^x}{\ln(a)}+c$ \hfil (für $a>0$ und $a\neq 1$)\\[0.75em]
    $\displaystyle\int \sin(x) \,\textrm{d}x$           & $\quad=\quad$ & $\displaystyle -\cos(x)+c$\\[0.75em]
    $\displaystyle\int \cos(x) \,\textrm{d}x$           & $\quad=\quad$ & $\displaystyle \sin(x)+c$\\[0.75em]
\end{tabular}

\hypertarget{rechenregeln}{%
\subsubsection{Rechenregeln:}\label{rechenregeln}}

\begin{description}
\tightlist
\item[Hauptsatz der Integral- und Differentialrechnung:]
\(\displaystyle \int_a^b f(x) \,\textrm{d}x =F(x)\bigg|_a^b = F(b)-F(a)\)
\item[Punktregel]
\(\displaystyle \int_a^a f(x) \,\textrm{d}x = 0\)
\item[Umkehrregel]
\(\displaystyle \int_a^b f(x) \,\textrm{d}x = -\int_b^a f(x) dx\)
\item[Linearität]
\(\displaystyle \int a f(x) + b g(x) \,\textrm{d}x = a \int f(x) dx + b \int g(x) dx\)
\item[Partielle Integration]
\(\displaystyle \int f(x) g'(x) \,\textrm{d}x = f(x)g(x) - \int f'(x)g(x) dx\)
\item[Substitionsregel für lineare Substitution]
Sei \(z=mx+d\) eine lineare Substitution und \(F(z)\) die Stammfunktion
von f(z), dann gilt:

\(\displaystyle \int_a^b f(mx+d) \,\textrm{d}x = \frac{1}{m} \int_{m\cdot a+d}^{m\cdot b+d}f(z)\,\textrm{d}z\)
\end{description}

%
% AfterBodyInclude
% ================
%
%
% ---------------------------------------------------------------------------
\end{multicols}
% ---------------------------------------------------------------------------------------

\end{document}
