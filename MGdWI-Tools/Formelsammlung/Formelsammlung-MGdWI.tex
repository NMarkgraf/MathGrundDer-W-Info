% Options for packages loaded elsewhere
\PassOptionsToPackage{unicode}{hyperref}
\PassOptionsToPackage{hyphens}{url}
%
\documentclass[
  ngerman,
  a4paper,
  landscape, fontsize=9pt, version=first, enabledeprecatedfontcommands,
  DIV=6]{scrartcl}
\usepackage{lmodern}
\usepackage{amssymb,amsmath}
\usepackage{ifxetex,ifluatex}
\ifnum 0\ifxetex 1\fi\ifluatex 1\fi=0 % if pdftex
  \usepackage[T1]{fontenc}
  \usepackage[utf8]{inputenc}
  \usepackage{textcomp} % provide euro and other symbols
\else % if luatex or xetex
  \usepackage{unicode-math}
  \defaultfontfeatures{Scale=MatchLowercase}
  \defaultfontfeatures[\rmfamily]{Ligatures=TeX,Scale=1}
\fi
% Use upquote if available, for straight quotes in verbatim environments
\IfFileExists{upquote.sty}{\usepackage{upquote}}{}
\IfFileExists{microtype.sty}{% use microtype if available
  \usepackage[]{microtype}
  \UseMicrotypeSet[protrusion]{basicmath} % disable protrusion for tt fonts
}{}
\makeatletter
\@ifundefined{KOMAClassName}{% if non-KOMA class
  \IfFileExists{parskip.sty}{%
    \usepackage{parskip}
  }{% else
    \setlength{\parindent}{0pt}
    \setlength{\parskip}{6pt plus 2pt minus 1pt}}
}{% if KOMA class
  \KOMAoptions{parskip=half}}
\makeatother
\usepackage{xcolor}
\IfFileExists{xurl.sty}{\usepackage{xurl}}{} % add URL line breaks if available
\IfFileExists{bookmark.sty}{\usepackage{bookmark}}{\usepackage{hyperref}}
\hypersetup{
  hidelinks,
  pdfcreator={LaTeX via pandoc}}
\urlstyle{same} % disable monospaced font for URLs
\usepackage[left=1cm, right=1cm, top=1cm, bottom=1cm, landscape]{geometry}
\setlength{\emergencystretch}{3em} % prevent overfull lines
\providecommand{\tightlist}{%
  \setlength{\itemsep}{0pt}\setlength{\parskip}{0pt}}
\setcounter{secnumdepth}{-\maxdimen} % remove section numbering
%
% ZusatzPakete
% ============
%
%  Die folgende Pakete werden noch benötigt:
%
% ---------------------------------------------------------------------------
%\usepackage{systeme}
\usepackage{booktabs}
\usepackage{xspace}
\usepackage{mathtools}
\usepackage{array}
\usepackage[utf8]{inputenc}
%\usepackage[ngerman]{babel}
\usepackage{multicol}
\usepackage{amsmath}
\usepackage{amsfonts}
\usepackage{amssymb}
\usepackage{gensymb}
\usepackage{dsfont}
\usepackage{calc}
\usepackage{csquotes}

\usepackage{enumitem}
\usepackage{xcolor}
\definecolor{FOMVoll}{RGB}{0,153,138}
\newcommand{\mydesrcitem}[1]{\textcolor{FOMVoll!90}{#1}}
%\setlist[description]{style=nextline,font=\sffamily\mydesrcitem}
\setlist[description]{style=nextline,font=\normalfont\mydesrcitem,itemsep=0.8em}

\DeclareTextFontCommand{\textbf}{\bfseries\color{red}}
\DeclareTextFontCommand{\emph}{\bfseries\color{blue}}

%\usepackage[permil]{overpic} \usepackage{graphicx} \graphicspath{{gfx/}
% ---------------------------------------------------------------------------------------
%
% DeclareOwnMathOperators
% =======================
%
%  Ein paar eigene mathematische Operatoren definieren
%
% ---------------------------------------------------------------------------
\DeclareMathOperator*{\leftlim}{\text{l-\!}\lim} 		% l-lim
\DeclareMathOperator*{\rightlim}{\text{r-\!}\lim} 	% r-lim
\DeclareMathOperator*{\grad}{\text{grad}} 			% grad
\DeclareMathOperator*{\Pot}{\mathcal{P}}
% ---------------------------------------------------------------------------------------
\ifxetex
  % Load polyglossia as late as possible: uses bidi with RTL langages (e.g. Hebrew, Arabic)
  \usepackage{polyglossia}
  \setmainlanguage[]{german}
\else
  \usepackage[shorthands=off,main=ngerman]{babel}
\fi

\date{}

\begin{document}

% ---------------------------------------------------------------------------
%
% BeforeBodyInclude
% =================
%
%
% ---------------------------------------------------------------------------

%{\Huge\sffamily\textcolor{FOMVoll}{{\bfseries Formelsammlung} zur Klausur \enquote{Mathematik für Wirtschaftsinformatiker}}}
{\Huge\sffamily\textcolor{FOMVoll}{{\bfseries Formelsammlung} zur Klausur \enquote{Mathematische Grundlagen der (Wirtschafts-)Informatik}}}

\setlength{\columnsep}{0.75cm}
\begin{multicols}{3}
% ---------------------------------------------------------------------------

\hypertarget{notationen}{%
\subsection{Notationen}\label{notationen}}

\begin{description}
\tightlist
\item[Summenzeichen]
\(\displaystyle \sum\limits_{k=m}^n a_k = a_m + a_{m+1} + a_{m+2} + \dotsc + a_{n-1} + a_{n}\)
\item[Produktzeichen]
\(\displaystyle \prod\limits_{k=m}^n a_k = a_m \cdot a_{m+1} \cdot a_{m+2} \cdot \dotsc \cdot a_{n-1} \cdot a_{n}\)
\item[Fakultät]
\(\displaystyle n! = \prod\limits_{k=1}^n k = 1 \cdot 2 \cdot \dotsc \cdot (n-1) \cdot n\)

\(\displaystyle 0! = 1\)
\end{description}

\hypertarget{einfaches-rechnen}{%
\subsection{Einfaches Rechnen}\label{einfaches-rechnen}}

\begin{description}
\item[Betrag]
Für eine reelle Zahl \(x\) ist der \textbf{(Absolut-)Betrag} definiert
durch:

\(\displaystyle |x| = \sqrt{x^2} = \begin{cases} \phantom{-}x &: x > 0 \\ \phantom{-}0 &: x=0 \\ -x &: x <0 \end{cases}\)
\item[Rechnen mit Beträgen]
Für reelle Zahlen \(x\),\(y\) und eine nicht-negative reelle Zahl \(p\)
gelten die folgenden Regeln:

\begin{tabular}{ l c l }
    $|x| \geq 0$                                    & $\qquad$  &   $|x| = 0 \Longleftrightarrow x=0$ \\
    $|x \cdot y| = |x| \cdot |y|$                   &   & \\
    $|x \cdot p| = |x| \cdot p$                     & $\qquad$  &   $|x \cdot (-p)| = |x| \cdot p$ \\
    $|x+y| \leq |x| + |y|$                          & $\qquad$  &   $|x-y| \geq \left| |x| - |y| \right|$ \\
    $\left|\frac{x}{y}\right| = \frac{|x|}{|y|}$    & $\qquad$  &   \\
\end{tabular}
\item[Bruchrechnen]
Für alle Zahlen \(a\), \(b\), \(c\), \(d\) mit \(c\neq 0\) und
\(d \neq 0\) gilt:

\begin{tabular}{ l c l }
    $\displaystyle \frac{a}{c} + \frac{b}{d} = \frac{ad+bc}{cd}$    & $\qquad$  &   $\displaystyle \frac{a}{c} - \frac{b}{d} = \frac{ad-bc}{cd}$ \\
    $\displaystyle \frac{c\cdot a}{c \cdot d} = \frac{a}{d}$        & $\qquad$  &   $\displaystyle \frac{a}{c} \cdot \frac{b}{d} = \frac{ab}{cd}$ \\
    $\displaystyle \frac{\frac{a}{c}}{\frac{b}{d}} = \frac{ad}{bc}$ & & \\
\end{tabular}
\item[Potenzrechengesetze]
Für reelle Zahlen \(a\neq0\) und \(b\neq0\), reelle Zahlen \(r\) und
\(s\) falls \(a>0\) und rationale Zahlen \(r\) und \(s\) falls \(a<0\)
ist gilt:

\begin{tabular}{ l c l }
    $\displaystyle a^0 = 1$                 & $\qquad$  & $\displaystyle a^{-r} = \frac{1}{a^r}$ \\
    $\displaystyle a^{r+s} = a^r \cdot a^s$ & $\qquad$  & $\displaystyle a^{r-s} = \frac{a^r}{a^s}$ \\
    $\displaystyle (a \cdot b)^{r} = a^r \cdot b^r$ & $\qquad$  & $\displaystyle \left(\frac{a}{b}\right)^{r} = \frac{a^r}{b^r}$ \\
    $\displaystyle (a^r)^{s} = a^{r \cdot s}$ & & \\
\end{tabular}

Für positive Zahlen \(a\) kann man die Potenz durch Exponentialfunktion
und Logaritmus ausdrücken:

\(\displaystyle x^{r} = \exp\left(r \cdot \ln(x)\right)\)
\item[Wurzelrechnengesetze]
Für positive Zahlen \(a\) und \(b\) und \(n,m,k \in \mathbb{N}\) gilt:

\begin{tabular}{ l c l }
$\displaystyle \sqrt[n]{a}\cdot\sqrt[n]{b}=\sqrt[n]{a\cdot b}$ & & 
$\displaystyle \frac{\sqrt[n]{a}}{\sqrt[n]{b}}=\sqrt[n]{\frac{a}{b}}$ \\
$\displaystyle \sqrt[k]{\sqrt[n]{a}}=\sqrt[k\cdot n]{a}$ & & 
$\displaystyle a^{\frac{m}{n}}=\sqrt[n]{a^m}=\left(\sqrt[n]{a} \right)^m$ \\
$\displaystyle a^{-\frac{m}{n}}=\frac{1}{a^\frac{m}{n}}$ & &
$\displaystyle \sqrt[n]{a}\cdot\sqrt[m]{a}=a^{\frac{1}{n}+\frac{1}{m}}=\sqrt[nm]{a^{n+m}}$ \\
\end{tabular}

Höhere Wurzeln aus positiven Zahlen \(x\) kann man wie jede Potenz durch
Exponentialfunktion und Logarithmus ausdrücken:

\(\displaystyle \sqrt[n]{x} = x^{1/n} = \exp\left(\frac{\ln(x)}{n}\right)\)
\item[Logarithmengesetze]
Für reellen, positive Zahlen \(a,b, x, y\) mit \(a, b \neq 1\), einem
reellen \(r\) und einer natürlichen Zahl \(n\) gilt:

\begin{tabular}{ l c l }
    $\displaystyle \log_a(1) = 0$ & & \\
\end{tabular}

\begin{tabular}{ l c l c l}
    $\displaystyle \text{lb}(x) = \log_2(x)$ & $\,$ & $\displaystyle \ln(x) = \log_e(x)$ & $\;$ & $\displaystyle \text{lg}(x) = \log_{10}(x)$ \\
\end{tabular}

\begin{tabular}{ l }
    $\displaystyle \log_a (x \cdot y) = \log_a(x) + \log_a(y)$ \\
    $\displaystyle \log_a \left(\frac{x}{y}\right) = \log_a(x) - \log_a(y)$ \\
    $\displaystyle \log_a(x^r) = r \cdot \log_a(x)$ \\
    $\displaystyle \log_a\left(\frac{1}{x}\right) = - \log_a(x)$ \\
    $\displaystyle \log_a(x + y) = \log_a(x) + \log_a\left(1+ \frac{x}{y}\right)$ \\
    $\displaystyle \log_b\left(\sqrt[n]{x}\right) = \log_b \left(x^{\frac 1n}\right) = \frac 1n\log_b x$ \\
    $\displaystyle \log_a(x) = \frac{\log_b(x)}{\log_b(a)}$
\end{tabular}
\item[Binomische Formeln]
Für reelle Zahlen \(x\) und \(y\) gelten die folgenden Regeln:

\(\displaystyle (x+y)^2 = x^2 + 2 xy + y^2\)

\(\displaystyle (x-y)^2 = x^2 - 2 xy + y^2\)

\(\displaystyle (x-y)(x+y) = x^2-y^2\)
\item[Binomischer Lehrsatz]
Für zwei reelle Zahlen \(x\), \(y\) und eine natürliche Zahl \(n\) gilt:

\(\displaystyle (x+y)^n = \sum\limits_{k=0}^{n} \dbinom{k}{n} x^{n-k}y^{k}\)
\item[Normalform von Polynomgleichungen]
Jede Polynomgleichung (2. Grades) der Form \(a x^2+ bx +c = d\), mit
\(a\neq 0\) lässt sich umformen in \textbf{Normalform} der Art
\(x^2+px+q = 0\).
\item[Diskriminante]
Für eine Polynomgleichung (2. Grades) ist die \textbf{Diskriminante}
definiert durch \(D=\frac{p^2-4\cdot q}{4}\).

Es gilt:

\begin{itemize}
\tightlist
\item
  \(D < 0\): die Gleichung hat keine (reelle) Lösung!
\end{itemize}

\begin{itemize}
\tightlist
\item
  \(D = 0\): die Gleichung hat eine Lösung nämlich \(-\frac{p}{2}\).
\end{itemize}

\begin{itemize}
\tightlist
\item
  \(D > 0\): die Gleichung hat zwei Lösungen. (\(\rightarrow\)
  pq-Formel)
\end{itemize}
\item[\(pq\)-Formel]
Für eine Polynomgleichung (2. Grades) mit positiver Diskriminante findet
sich die Nullstellen \(x_{1/2}\) durch

\(\displaystyle x_{1/2} = -\frac{p}{2} \pm \sqrt{D} = -\frac{p}{2} \pm \sqrt{\left(\frac{p}{2}\right)^2 - q}\)
\item[Satz von Vieta]
Für die Lösungen \(x_1\) und \(x_2\) einer Polynomgleichung (2. Grades)
in Normalform gilt:

\(x_1 \cdot x_2 = q\) und \(-(x_1+x_2)=p\)
\end{description}

\hypertarget{logik}{%
\subsection{Logik}\label{logik}}

\begin{description}
\tightlist
\item[Aussagen]
Sätze, die entweder \emph{wahr} oder \emph{falsch} sind, heißen
\textbf{Ausagen}.
\item[Aussageformen / offene Aussagen]
Hängte die Wahrheit einer Aussage von einem Parameter \(x\) ab, so nennt
man die Aussage \(A(x)\) eine \textbf{offene Aussage} oder
\textbf{Aussageform}.
\item[Lösungsmenge]
Die Menge der Werte \(x\), die eine Aussageform \(A(x)\) zu einer
\emph{wahren Aussage} machen heißt \textbf{Lösungemenge}
\end{description}

Es seien \(A\) und \(B\) Aussagen, dann gilt:

\begin{description}
\tightlist
\item[Implikation \emph{(Aus \(A\) folge \(B\))}]
\(\displaystyle A \Longrightarrow B\) : falls \(A\) wahr ist, dann ist
auch \(B\) wahr.
\item[Äquivalenz]
\(\displaystyle A \Longleftrightarrow B\) : \(A\) ist genau dann wahr,
falls \(B\) wahr ist.
\item[Konjunktion]
\(\displaystyle A \wedge B\) : \(A\) ist wahr und \(B\) ist wahr.
\item[Disjunktion]
\(\displaystyle A \vee B\) : \(A\) ist wahr oder \(B\) ist wahr.
\item[Negation]
\(\displaystyle \lnot A\) ist wahr \(\Longleftrightarrow\) \(A\) ist
falsch.
\item[Allquantor]
\(\displaystyle \forall\) : \enquote{Für alle}"
\item[Existenzquantor]
\(\displaystyle \exists\) : \enquote{Es gibt ein}
\end{description}

\hypertarget{mengenlehre}{%
\subsection{Mengenlehre}\label{mengenlehre}}

Für beliebige Mengen \(A\) und \(B\) gilt:

\begin{description}
\tightlist
\item[Element]
Ist \(a\) ist ein \textbf{Element} von \(A\), dann schreiben wir
\(a \in A\).
\item[Teilmenge]
\(\displaystyle A \subseteq B \Longleftrightarrow \left(x \in A \Rightarrow x \in B\right)\)
\item[Echte Teilmenge]
\(\displaystyle A \subsetneq B \Longleftrightarrow \left(A \subset B \wedge \exists z \in B : z \notin A\right)\)
\item[Gleichheit von Mengen]
\(\displaystyle A = B \Longleftrightarrow A \subseteq B \wedge B \subseteq A\)
\item[Vereinigungsmenge zweier Mengen]
\(\displaystyle A \cup B = \{ x \mid x \in A \vee x \in B \}\)
\item[Schnittmenge zweier Mengen]
\(\displaystyle A \cap B = \{ x \mid x \in A \wedge x \in B \}\)
\item[Kompliment einer Menge]
\(\displaystyle A^c = \{x \mid x \in U \wedge x \not\in A \}, U\) ein
\emph{Universum} mit \(A \subset U\)
\item[Differenz von Mengen]
\(\displaystyle A \setminus B = \{ x \mid x \in A \wedge x \notin B \} = A \cap B^c\)
\item[Gleichmächtigkeit von Mengen]
\(A\) und \(B\) sind gleichmächtig, falls es eine Bijektion
\(f: A \leftrightarrow B\) gibt.
\item[Endlichkeit]
Eine Menge ist \textbf{endlich}, wenn sie \emph{gleichmächtig} zu einem
Element von \(\mathbb{N}_0\) im Sinne von \{\it von Neumann\} ist.
\item[Abzählbar]
Eine Menge ist \textbf{abzählbar}, wenn sie \emph{endlich} ist oder
\emph{gleichmächtig} zu einer \emph{Teilmenge} von \(\mathbb{N}\) ist.
\item[Unendlichkeit]
Eine nicht \emph{endliche} Menge ist \textbf{unendlich}
\item[Mächtigkeit von Mengen (allgemein)]
\(|A|\) heißt \emph{Betrag} der Menge \(A\) und bezeichnet die
Mächtigkeit der Menge.
\item[Mächtigkeit von endlichen Mengen]
\(|A|\) ist die Anzahl der unterscheidbaren Elemente der (endlichen)
Menge \(A\).
\item[Potenzmenge]
\(\displaystyle \Pot(A) = \{ U \mid U \subset A\}\)
\item[Satz von Cantor]
Für jede Menge \(A\) gilt: \(|A| < |\Pot(A)|\)
\item[Produktmenge]
\(\displaystyle A \times B = \{(x;y) \mid x \in A \wedge y \in B\}\)
\item[De Morgansche Regeln]
\(\displaystyle (A \cup B)^c = A^c \cap B^c\) und
\(\displaystyle (A \cap B)^c = A^c \cup B^c\)
\item[Disjunktheit]
\(A\) und \(B\) sind \emph{disjunkt}
\(\displaystyle\Longleftrightarrow A \cap B = \emptyset\)
\item[Zerlegung / Partition]
Die Mengen \(A_1, ..., A_n\) mit
\(A_1 \cup A_2 \cup \cdots \cup A_n = A\) und
\(A_i \cap A_j = \emptyset\) für alle \(0 \leq i \not= j \leq n\) heißt
\emph{Partition} oder \emph{Zerlegung} von \(A\).
\end{description}

\hypertarget{zahlen}{%
\subsection{Zahlen}\label{zahlen}}

\begin{description}
\tightlist
\item[Natürliche Zahlen]
\(\displaystyle \mathbb{N} = \{1,2,3,4,...\}\)
\item[Natürliche Zahlen mit Null:]
\(\displaystyle \mathbb{N}_0 = \mathbb{N} \cup \{0\} = \{0,1,2,3,4,...\}\)
\item[Ganze Zahlen]
\(\displaystyle \mathbb{Z} = \{...,-3,-2,-1,0,1,2,3,...\}\)
\item[Rationale Zahlen]
\(\displaystyle \mathbb{Q} = \left\{ \left. \frac{q}{p} \right|\,q\in \mathbb{Z}, p \in \mathbb{N}, p \text{ und } q \text{ sind teilerfremd}\right\}\)
\item[Reelle Zahlen]
\(\displaystyle \mathbb{R}\)
\item[Komplexe Zahlen]
\(\displaystyle \mathbb{C} = \left\{ x + y\cdot i \left| \,x,y \in \mathbb{R}\right. \right\}\)
\end{description}

Es gilt:
\[\mathbb{N} \subsetneq  \mathbb{N}_0 \subsetneq  \mathbb{Z} \subsetneq \mathbb{Q} \subsetneq \mathbb{R}\subsetneq \mathbb{C}\]

\hypertarget{vollstuxe4ndige-induktion}{%
\subsection{Vollständige Induktion}\label{vollstuxe4ndige-induktion}}

Sei \(A(n)\) eine Aussageform, die es für alle \(n \in \mathbb{N}\) zu
beweisen gilt

\begin{itemize}
\tightlist
\item
  \textbf{Induktionsanfang:} \(A(1)\) gilt.
\item
  \textbf{Induktionsschritt:} Unter der Annahme das \(A(n)\) gilt zeigt
  man, dass \(A(n+1)\) gilt.

  \begin{itemize}
  \tightlist
  \item
    \textbf{Induktionsannahme}: Es gelte \(A(n)\).
  \item
    \textbf{Induktionsschluss}: Zu zeigen ist dann, dass \(A(n+1)\)
    gilt.
  \end{itemize}
\end{itemize}

\hypertarget{kombinatorik}{%
\subsection{Kombinatorik}\label{kombinatorik}}

\begin{description}
\tightlist
\item[Summenregel]
\(\displaystyle |A \cup B| = |A| + |B| - |A \cap B|\)
\item[Inklusion und Exklusion]
\(\displaystyle |A \cup B \cup C| = |A| + |B| + |C| - |A \cap B| - |A \cap C| - |B \cap C| + |A \cap B \cap C|\)
\item[Produktregel]
\(\displaystyle |A \times B| = |A| \cdot |B|\)
\item[\(k\)-Permutationen / Variation]
\(\displaystyle P(n, k) = n \cdot (n-1) \cdot (n-2) \cdot \dotsc \cdot (n-k+1) = \frac{n!}{(n-k)!}\)
\item[Permutation]
\(\displaystyle n! = P(n, n) = n \cdot (n-1) \cdot (n-2) \cdot \dotsc \cdot 1\)
\item[Binomialkoeffizient]
\(\displaystyle \dbinom{n}{k} = C(n,k) = \frac{P(n,k)}{k!} = \frac{n!}{k! \cdot (n-k)!}\)
\end{description}

Für die Anzahl der Möglichkeiten aus \(n\) Objekten \(k\) Objekte
auszuwählen, gelten die folgenden Regeln:

\begin{center}
    \begin{tabular}{lcc}
        \toprule
        Auswahl & \textbf{mit} Beachtung    & \textbf{ohne} Beachtung   \\
        ~       & der Reihenfolge           & der Reihenfolge           \\
        ~       & (\textit{Variation})      & (\textit{Kombination})    \\
        \midrule
        \textbf{ohne} Zurücklegen & $\displaystyle\frac{n!}{(n-k)!}$    & $\displaystyle\dbinom{n}{k}$ \\
        \midrule
        \textbf{mit}  Zurücklegen & $\displaystyle n^k$             & $\dbinom{n+k-1}{k}$ \\
        \bottomrule
    \end{tabular}
\end{center}

\hypertarget{lineare-algebra}{%
\subsection{Lineare Algebra}\label{lineare-algebra}}

\begin{description}
\item[Lineares Gleichungssystem]
Ein LGS mit \(m\) Gleichungen und \(n\) unbekannten Variabeln hat die
Form :\[\begin{matrix}
    a_{11} \cdot x_1 +  a_{12} \cdot x_2 \, + & \cdots & +\, a_{1n} \cdot x_n & = & b_1\\
    a_{21} \cdot x_1 +  a_{22} \cdot x_2 \, + & \cdots & +\, a_{2n} \cdot x_n & = & b_2\\
    &&&\vdots&\\
    a_{m1} \cdot x_1 +  a_{m2} \cdot x_2 \, + & \cdots & +\, a_{mn} \cdot x_n & = & b_m\\
\end{matrix}\]

\(a_{ij}\) : Koeffizienten

\(b_i\) : rechte Seite
\item[Homogene / Inhomogene LGS]
Sind alle \(b_i=0\), nennt man das LGS \textbf{homogen}, sonst
\textbf{inhomogen}

Homogene LGS besitzen immer eine \textbf{triviale Lösung}, bei der alle
\(x_i=0\) sind.
\item[Quandratische LGS]
Ist \(m=n\) so nennt man das LGS \textbf{quadratisch}
\item[Elementare Zeilenumformungen]
Man ändert die Lösungsmenge eines LGS nicht, wenn man

\begin{itemize}
\tightlist
\item
  zwei Zeilen vertauscht,
\end{itemize}

\begin{itemize}
\tightlist
\item
  eine Zeile auf beiden Seiten mit einer beliebigen Konstante
  \(c \neq 0\) multipliziert,
\end{itemize}

\begin{itemize}
\tightlist
\item
  das Vielfache einer Zeile zu einer anderen hinzuaddiert oder
\end{itemize}

\begin{itemize}
\tightlist
\item
  das Vielfache einer Zeile von einer anderen subtrahiert.
\end{itemize}
\item[Eliminationsverfahren]
Man benutzt die \emph{elementaren Zeilenumformungen} um aus einem
beliebigen LGS ein LGS in \textbf{Zeilenstufenform} oder
\textbf{Diagonalgestallt} zu erhalten. Das Ziel ist dabei die Lösungen
einfach oder gar direkt abzulesen.
\item[Lösungsverhalten eines LGS]
Ein \textbf{homogenes} LGS besitzt entweder

\begin{itemize}
\tightlist
\item
  genau \textbf{eine} Lösung, nämlich die triviale Lösung oder
\end{itemize}

\begin{itemize}
\tightlist
\item
  \textbf{unendlich viele} Lösungen.
\end{itemize}

Ein \textbf{inhomogenes} LGS besitzt entweder

\begin{itemize}
\tightlist
\item
  genau \textbf{eine} Lösung oder
\end{itemize}

\begin{itemize}
\tightlist
\item
  \textbf{unendlich viele} Lösungen oder
\end{itemize}

\begin{itemize}
\tightlist
\item
  überhaupt \textbf{keine} Lösung.
\end{itemize}
\item[Matrizen]
\textbf{Matrizen} sind geordnete, rechteckige Schemata von Zahlen oder
Symbolen. \[\renewcommand{\arraystretch}{.9}
A=\begin{pmatrix}
a_{11} & a_{12} & \ldots & a_{1j} &\ldots& a_{1n}\\
a_{21} & a_{22} & \ldots & a_{2j} & \ldots & a_{2n}\\
\vdots & \vdots & & \vdots & &\vdots \\
a_{i1} & a_{i2} & \ldots & a_{ij} & \ldots & a_{in}\\
\vdots & \vdots & & \vdots & &\vdots \\
a_{m1} & a_{m2} & \ldots & a_{mj} & \ldots & a_{mn}\\
\end{pmatrix}
   = {(a_{ij})}_{m\times n}\]

mit \(m\) und \(n \in \mathbf{N}\).

\(m\) : Zeilen

\(n\) : Spalten

\(m\times n\) : Orndung der Matrix

\(a_{11},\ldots,a_{mn}\) : Elemente der Matrix

\(i\) : Zeilenindex

\(j\) : Spaltenindex
\item[Vektoren]
\(n\times 1\)-Matrix heißt \textbf{Spaltenvektor mit \(n\) Komponenten}

\(1\times n\)-Matrix heißt \textbf{Zeilenvektor mit \(n\) Komponenten}
\item[Skalar]
Einen Wert aus dem Grundkörper (meistens \(\mathbb{R}\)) nennen wir
einen \textbf{Skalar}.
\item[Addition \& Subtraktion von Matrizen und Vektoren]
Die \textbf{Addition} und \textbf{Subtraktion} von Matrizen gleicher
Ordnung erfolgt \textbf{komponentenweise}.
\item[Multiplikation mit einem Skalar]
\emph{Matrix} werden mit einem \emph{Skalar} multiplizieren, in dem wir
\emph{jedes Element} mit dem \emph{Skalar multipliziert}.
\item[Linearkombination]
\(v_1, \dots v_n, v\) Vektoren. \(v\) ist \textbf{Linearkombination},
falls gilt: \[v = \sum_{i=1}^{n} c_i v_i\]
\item[Lineare (Un-)abhängigkeit]
Eine Menge von Vektoren ist \textbf{linear unabhängig} falls keiner von
ihnen als \emph{Linearkombination} der anderen ausgedrückt werden kann.

Ansonsten sind sie \emph{linear abbhängig}.
\item[Multiplikation von Matrizen]
Sei \(A_{n \times p}\), \(B_{p \times m}\), dann lässt sich
\(C_{n \times m} =A \cdot B\) berechnen mit
\[c_{ij} = \sum_{k=1}^{p} a_{ik}\cdot b_{kj}\]

für \(1 \leq i \leq n\) und \(1 \leq j \leq m\).
\item[Transposition]
Die transponierte Matrix \(A^T\) einer Matrix \(A\) ergibt sich in dem
jede Spalte von \(A\), bei gleichbleibender Reihenfolge, zu einer Zeile
von \(A^T\) wird.
\item[Skalarprodukt]
Das \textbf{Skalarprodukt} zweier (Spalten-)Vektoren \(x\) und \(y\)
lautet: \[<x, y> = x^T \cdot y\] Einheitsmatrix

\(E_n\) heißt \textbf{Einheitsmatrix} mit \(n \times n\) Elementen, wenn
gilt: \[e_{ij} = \begin{cases}
     1 & i=j \\ 
     0 & i\neq j 
\end{cases}\]
\item[Inverse einer Matrix]
Gibt es zu \(A_{n \times n}\) eine Matrix \(X\) mit
\[E_n = X \cdot A = A \cdot X = E_n\]

so nennen wir \(X\) die \textbf{Inverse der Matrix \(A\)} und schreiben
dafür \(A^{-1}\).
\end{description}

\hypertarget{finanzmathematik}{%
\subsection{Finanzmathematik}\label{finanzmathematik}}

\begin{description}
\tightlist
\item[Notationen]
\(K\), \(K_0\), \(K_t\) : Kapital (zum Zeitpunkt 0 oder t)

\(t\) : Zeitpunkt oder Zeitraum

\(Z\),\(Z_t\) : Zinsen(für den Zeitraum \(t\))

\(i\) : Zins, Zinssatz

\(q=1+i\) : Aufzinsungsfaktor
\item[Zinseszinsformel]
\(K_n = K_0 \cdot (1+i)^n = K_0 \cdot q^n\)
\item[Unterjährige Verzinsung]
\(K_t = K_0 + Z \cdot t = K_0 + i\cdot K_0 \cdot t = K_0 (1+i\cdot t)\)

\(t = \frac{T_2-T_1}{360}\)

\(T_2\) : Auszahlungszeitpunkt in Zinstagen

\(T_1\) : Einzahlungszeitpunkt in Zinstagen

\(T_i = (\text{aktueller Monat} - 1) \cdot 30 + \text{Tag im Monat}\)
\item[Gemischte Verzinsung]
\(K_t = K_0\cdot \left(1+i\cdot t_1\right)\cdot \left(1+i\right)^n\cdot \left(1+i\cdot t_2\right)\)

\(K_t = K_0\cdot \left(1+i\cdot \frac{360-T_0+1}{360}\right)\cdot \left(1+i\right)^n\cdot \left(1+i\cdot \frac{T_1-1}{360}\right)\)

\(T_0\) : Einzahlungszeitpunkt in Zinstagen im ersten Jahr

\(t_0\) : Anlagedauer in Zinstagen im ersten Jahr

\(T_1\) : Auszahlungszeitpunkt in Zinstagen im letzten Jahr

\(t_1\) : Anlagedauer in Zinstagen im letzten Jahr

\(n\) : Anzahl der ganzen Jahre
\item[Approximative Verzinsung]
\(K_t = K_0 \cdot (1+i)^t = K_0 \cdot q^t\)

\(t\) : Anlagedauer als nicht-ganzzahliger Wert
\end{description}

%
% AfterBodyInclude
% ================
%
%
% ---------------------------------------------------------------------------
\end{multicols}
% ---------------------------------------------------------------------------------------

\end{document}
